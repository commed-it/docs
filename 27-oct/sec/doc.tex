\begin{document}
\let\texteuro\euro
\subsection{Monezitation Strategy}
The monetization strategy of Commed is mainly based on a percentage of the value of the contracts. When a contract has been finally set, an invoice will be generated. The value of the invoices depends on the value of the generated contract in order to make the invoices affordable for all types and sizes of companies. Therefore, in the first four years, the search of the companies and the other features will be free. After these beginning years, it is something to study and plan if some other features can become freemium.
\\
\\
In order to get more publications in the first few years and populate our service quickly, the first three publications will be free to any company. The next ones will cost 250 € per year, which will not be depending on the size of the company publishers.
In addition, any kind of publication could get a better position in the search by paying monthly. The fee of this kind of advertising will be up to 450 € per month and the position will depend on how much the company pays for that advertising.
\\
\\
Therefore, there are three possible incomes:
\begin{itemize}
	\item 5\% commission of each contract generated.
	\item Publish a 4\supindex{th}, 5\supindex{th},..., n\supindex{th} offer cost 250 € per new offer each year. \\Publishers have 3 free publications.
	\item Payments for advertising by month, they payment will be up to the customer, such as instagram and facebook. The more the company pays for the advertising, the more will appear the offer. The maximum fee should be 450 € per month.
\end{itemize}
As the search will be free, the companies could be able to negotiate and set the contract out of our reach. In order to sort this inconvenient out, Commed will finally generate the contract making this process so easy that they could afford the invoice as the payment is only the 5\% of the contract.
\subsection{Marketing Strategy}
According to the marketing strategy, it has been set that the main focus in the first years will be getting the small and medium companies, in order to quickly populate the platform and create small business ecosystems. For example, the main ecosystems to focus in the first year will be in order of importance:
\begin{itemize}
	\item Restauration, butchers and other food providers.
	\item Food industry and supermarkets.
	\item Organizations for seasonal work to get temporary contracts.
	\item Cleaning services.
	\item Security services.
\end{itemize}
After creating the small ecosystem, the focus will be on strengthening the ecosystem and widening the smaller ones in order to join with the others and make them become a greater one.
\subsection{Speculation and flow chart}
In order to create the simulation of the revenue, a speculation has been done on how many contracts will be held by our application in a month. This  conjecture has been simulated over the first four years of the company. The revenue functions consists on these three variables:
\begin{itemize}
	\item The income from the percentage of the contract between the entities. This variable has the main weight of the function as it is the main income generator.
	\item The revenue from the paid publications. In order to ease the simulation and know how many publications will be paid, a relaxation of the problem has been made. The paid publications ratio has been calculated with an approximation, which also is directly correlated with the amount of contracts held.
	\item The income from the payments for advertising a publication. This variable has also been approximated making a correlation from the number of contracts.
\end{itemize}
According to the costs, its function depends on these variables:
\begin{itemize}
	\item The cost of paying developers, which can be splitted into junior or senior developers.
	\item Marketing cost.
	\item As the platform will be growing, the costs of the server will be higher. Despite that, it is thought about creating a serverless backend using AWS Lambda to minimize this cost and only pay for the requests.
\end{itemize}
Three scenarios have been made in order to show a possible but different projection of the incomes generated by the platform:
\begin{itemize}
	\item A pessimistic scenario, where the number of contracts increases in a linear way amongst the four years.
	\item An optimistic scenario, in which the contracts’ function has an exponential form.
	\item A more realistic scenario, which also has linear function in the first years but in the lasts, it gets more the way of an exponential function.
\end{itemize}
In Figure \ref{fig:cashflow}, we can see the contracts’ function of all the scenarios and the total cost function, which is the same in all scenarios. In the optimistic scenario, we will start to recover the initial investment at the beginning of the first year whereas in the worst case, we will be starting generating benefits at the beginning of the third year.
\begin{figure}
	\centering
	\includegraphics[width=15cm]{CashFlow.png}
	\caption{Cash flow of the different revenue function from scenarios and cost function}
	\label{fig:cashflow}
\end{figure}
\subsection{Economic indices comparison between scenarios}
All information related to the simulation and prediction about the first four years of the platform can be found in the spreadsheet attached to this document. Even though, the cash flows of each scenario can be found in \textit{Tables \ref{tab:pessimistic}, \ref{tab:realistic} and \ref{tab:optimistic}.}
In this section, we will begin by comparing the three scenarios using the different indexes calculated above. \\
\\
First, we want to show you the \texttt{ROI} index between the different scenarios, which can be found in Table \ref{tab:roi}. The \texttt{ROI} has been calculated every year, as for us it is more significant information to being analyzed. Although it can be found that the \texttt{ROI} index in the first year is similary between the scenarios, the pessimistic scenario has some challenges to increase the ROI so as to be positive while optimistic and, actually, the realistic appear to have a better return on investment.
\begin{table}[H]
	\centering
	\includegraphics[width=11cm]{roi.png}
	\caption{ROI Index comparison between the three scenarios}
	\label{tab:roi}
\end{table}
According to the other indices, mostly all of them have better values in optimistic scenario. One of the most valuable indices is the \texttt{PBP} in terms of months, where we can see that the optimistic has only 9 months of \texttt{PayBack Period}. Also, realistic scenario is a promising case, because in only one year and a month we would be able to recover the cost of the investment. Although the \texttt{ROI} in pessimistic scenario gave us bad information about the case, we can see that in all the scenarios the \texttt{IRR} is positive, which is a good information to know if someone is going to invest on the company. The only one index that has the same behaviour in all the cases it is the \texttt{BEP}, which is calculated monthly. That is because the costs are fixed, as the lifecycle of the software would be large.
\begin{table}[H]
	\centering
	\includegraphics[width=11cm]{indices.png}
	\caption{Indices comparison between the three scenarios}
	\label{tab:indicies}
\end{table}
\begin{table}[H]
	\centering
	\includegraphics[width=11.3cm]{pessimistic.png}
	\caption{Pessimistic Cash Flow}
	\label{tab:pessimistic}
\end{table}
\begin{table}[H]
	\centering
	\includegraphics[width=11.3cm]{realistic.png}
	\caption{Realistic Cash Flow}
	\label{tab:realistic}
\end{table}

\begin{table}[H]
	\centering
	\includegraphics[width=11.3cm]{optimistic.png}
	\caption{Optimistic Cash Flow}
	\label{tab:optimistic}
\end{table}
\end{document}