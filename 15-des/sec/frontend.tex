% !TeX spellcheck = en_US
\documentclass[./main.tex]{subfiles}
\begin{document}
\subsection{Reason on implementing a Web Client}
As the application has its target user in the 
administrative world, it was decided that 
a web will be created alongside the mobile application that will
make the usage of our services easier to our customers
in administrative environments. It has been decided that the 
mobile app can be really useful for doing little 
tramits, for chatting with some customers, for 
quick searches of products/services and enterprises...\\
But it has been felt that in order to make it more usable and
accessible for administrative porposes a web client
should also be created.
\\\\
Furthermore, the implementation of the web client 
shined some light on how to communicate with the backend
and has lead us to iterate a little bit on the API endpoints
that were implemented on the last sprint. This is nice because now
on the third sprint the integration of the mobile app with the 
backend will be a lot more straightforward.
\subsection{React}
For implementing this Web Client React has been used, which is a 
library built in javascript for creating User Interfaces. 
To be more precise React was used because it gives the opportunity
of working with Hooks. Hooks are functions that comprehend inside them inmutable
components with independent states, wich makes the code a lot more
clean and structured and simplifies the amount of chaos that has
to be dealt with usually when it's programmed with vanilla javascript.
\\\\
Also React has a huge community, and it has a lot of libraries that
have been key for making a lot more easier the programming task.
Some examples can be:
\begin{itemize}
	\item react-router-doom
	\item react-modal
	\item redux-react-session
	\item react-bootstrap
	\item ...
\end{itemize}
\subsection{Web Client Architecture}
By now This Web Client is still in a development phase, so for 
now the default testing server that React provides to us is being used.
Although, a nginx+docker is expected to be used when deploying our application
to a production enviroment.
\\\\
Right now the application is able to fully connect with the backend,
beeing able to Register, SignUp and interacting with all the different 
features that our application provides.
\subsection{Redux}
In this part of the project, redux has also been used. In fact a react library
called redux-react-session that builds a store to maintain the state of our session is being used.
Also it creates the correspondig Session Cookie to maintain the value of the authorization token
provided by our backend. This way, every time a request that needs authorization has to be made,
it will be possible to pass the value of the token stored in the browser cookie within the request .
\end{document}