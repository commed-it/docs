% !TeX spellcheck = en_US
\documentclass[./main.tex]{subfiles}
\begin{document}
\section{Usability}
In this section it shall be described how the usability tests were done in this sprint. Although the methodologies were set for it, and some users were tested  with the application, the conclusions about the data that was held for the application will be shown in the final document and not in this sprint.
\subsection{Objectives of the usability test}
The first step of any usability testing session is to set the goals. One practical technique that can be used is a variant of the methodology outlined by Michael Margolis of Google Ventures. Basically, this involves asking a series of questions (interview) to those involved in the creation of the application (including the development team) to explore key areas.
The answers obtained from these interviews will give us important insights into what we already know and what we would like to know.

\subsection{Tasks to be carried out during the test}
Once the objectives have been set, it is time to move on to the next step - setting tasks. Tasks are usually just one long sentence and should consist of interactions that need to be carried out by the test users:

\begin{itemize}
	\item Log in to the application.
	\item Log out.
	\item Searching for a product/service.
	\item Contact a company.
	\item Reading information about a company.
\end{itemize}

Instead of asking the user to perform a task and thus making them feel like they are being tested, tasks should be turned into scenarios. These provide more insight into why the participant is performing the task, and therefore more closely resemble the natural interactions a typical user will have with the application.
In this sense, the work scenarios to be established should be:

\begin{itemize}
	\item Realistic, actionable and without any hints on how to perform the steps.
	\item Ordered in a sequence that ensures a smooth flow of the testing session.
	\item Tied to one or more objectives.
\end{itemize}

\subsection{Test documents and forms}
There are a number of documents that we will need when performing usability testing. While the number of documents and their content may vary, the most common is to have the following:

\begin{itemize}
	\item Consent form.
	\item Test 
	\item Pre-Test
	\item Post-test questionnaire.
\end{itemize}

\subsection{The participants or users of the test}

The usability testing method for a mobile application is a user-oriented testing technique, i.e., it involves real users performing realistic tasks that the application aims to achieve. Although, testing with real users is more resource intensive. This realistic scenario tends to produce more accurate results.
Nielsen Norman recommends recruiting test participants who have been using their devices for at least 3 months. This would overcome the difficulties associated with using the device rather than the application itself.

There are several considerations to take into account when choosing participants for a usability test, as they should: 

\begin{itemize}
	
	\item Be representative for the users the application is aimed at (target users).
	\item Own a device whose operating system is the one on which the application is intended to run (in our case, it will run on the Android emulator).
	\item Be available at the time, place and frequency of usability testing.
	\item Be prepared to sign a usability test participation consent form.
\end{itemize}

\subsection{Usability test methodology}

There are two main methods for conducting usability testing of a mobile application. Each comes with its own set of advantages and disadvantages. These are:

\begin{itemize}
	\item Lab usability testing.
	\item Remote usability testing.
\end{itemize}

For the realization of this test, it was necessary to access the usability room but in our case the test has been done on our own computer using the following components. 
\begin{itemize}
	\item Recording software.
	\item Test site.
	\item Person responsible for the success of the test.
\end{itemize}


\subsection{The test procedure}{}
\begin{table}[H]
	\centering
\begin{tabular}{r| l}
	\hline
	Step & Time \\ \hline
	Welcome / signature of the consent form & 5minutes \\
	Pre-test Interview & 5 minutes \\ 
	Carry out test tasks & 30 minutes \\ 
	Post-test questionnaire & 5 minutes \\ 
	SUS(System Usability Scale) & 5 minutes \\ \hline
\end{tabular}
\end{table} 
It is important to remember that the final step, the post-test interview, is a process that takes place at the end of each testing session and involves analyzing the actions performed by the participant. As this interview is conducted with the participant, it provides insight into why a participant performs these types of actions. Thus, while the testing session may indicate potential problems, it is the debriefing and post-test interview that provide insight into why those problems occurred.

\end{document}