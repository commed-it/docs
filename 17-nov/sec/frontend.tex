% !TeX spellcheck = en_US
\documentclass[./main.tex]{subfiles}
\begin{document}
\subsection{Reason on implementing a Web Client}
As the application has his target user in the 
administrative world, we decided to implement 
a web alongside the mobile application that will
make easyer the usage of our servies to our customers
in administrative enviroments. We though that the 
mobile app can be a really usefull for doing little 
tramits, for chatting with some customers, for 
quick searches of products/services and enterprises...\\
But we felt that in order to make it more usable and
accessible for administrative porpouses we should
create a web client also.
\\\\
Furthermore, the implementation of the web client 
showed some light on how to communicate with the backend
and has lead us to iterate a little bit on the API endpoints
that were implemented last sprint. This is nice because now
on the third sprint the integration of the mobile app with the 
backend will be a lot more straightforward.
\\\\
\subsection{React}
For implementing this Web Client we used React, which is a 
library built in javascript for creating User Interfaces. 
To be more precise we used React becouse it gives the oportunity
of working with Hooks. Hooks are functions that comprehend inside him inmutable
components with independent states, wich makes the code a lot more
clean and structured and simplifyes the amount of chaos that we have
to deal usually when we program with vanilla javascript.
\\\\
Also React has a huge community, and it has a lot of libraries that
have been key for making a lot more easyer the programming task.
Some examples can be:
\begin{itemize}
	\item react-router-doom
	\item react-modal
	\item redux-react-session
	\item react-bootstrap
	\item ...
\end{itemize}
\subsection{Web Client Architecture}
By now This Web Client is still in a development phase, so for 
now we are using the default testing server that React provides to us.
Although, we expect to use a nginx+docker when deploying our application
to a production enviroment.
\\\\
Right now the application is able to fully connect with the backend,
beeing able to Register, SignUp and interacting with all the different 
features that our application provides.
\subsection{Redux}
In this part of the project, we also used redux. In fact we use a react library
called redux-react-session, that builds a store to mantain the state of our session.
Also it creates the correspondig Session Cookie to mantain the value of the authorization token
provided by our backend. This way, every time we have to make a request that needs authoritzation,
we will be able to pass within the request the value of the token stored in the browser cookie.
\end{document}